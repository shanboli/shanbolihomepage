% ********** Chapter 2 **********
\chapter{Requirement}
\label{sec:Requirement}

\lettrine[lines=3]{T}{his} chapter contains the requirements for \textsf{Web Call Example Application}, from the aspect of programming language, stability, reusability, extendibility and integration. 

\section{Programming Language}
\label{sec:Requirement:ProgrammingLanguage}

To make the application potable, \textbf{Java} is chosen as the programming language of Web Call Example Application. The Java programming language is a popular high-level language which provides a portable feature and can be used on many different operating systems. For the introduction and more detail of Java please refer to \ref{sec:Java}.

\section{Stability}
\label{sec:Requirement:Stability}

The application should be stable and has as less bugs as possible. The application should be designed for deploying on a server for long term use. The concurrent request users may more than one hundred. 

\section{Reusability}
\label{sec:Requirement:Reusability}

The code should be made as generic and reusable. The interface should not constrain on any specific network or service provider. It should follow a common accepted standard. Session Initiation Protocol (SIP) is a signaling protocol, which defined in RFC 3261 SIP: Session Initiation Protocol \cite{RFC3261}, widely used for multimedia communication sessions such as voice and video calls over the Internet. It should be used as the main signal protocol of Web Call Example Application.

\section{Extendibility}
\label{sec:Requirement:Extendibility}
The application should be able to add new feature according to customer's requirement, e.g., add video call and instant message.

\section{Integration}
\label{sec:Requirement:Integration}

The Web Call Example Application should supply a web service API that can be used by other applications. This interface should contain most of the functions of Web Call Example Application. 

% ********** End of chapter **********
