% ********** Chapter 4 **********
\chapter{Solution}
\label{sec:Solution}

\section{Involved Technology}
\label{sec:Solution:InvolvedTechnology}

This section described all of the technologies that involved in the Web Call SDK project.

\subsection{Java}
\label{sec:Solution:InvolvedTechnology:Java}

The Java\texttrademark{} programming language is a popular high-level language which provides a portable feature and can be used on many different operating systems. 

The source code of Java is first written in plain text files which ends with the \nolinebreak\texttt{.java}. Then the Java source files are compiled into bytecodes which ends with \nolinebreak\texttt{.class} by Java compiler (javac). A \nolinebreak\texttt{.class} file is platform independent. It will be executed by the Java Virtual Machine\label{sym:JVM}\footnote{The terms ``Java Virtual Machine'' and "JVM" mean a Virtual Machine for the Java platform. \cite{TheJavaProgrammingLanguage}}.\cite{TheJavaProgrammingLanguage}

\subsection{SIP}
\label{sec:Solution:InvolvedTechnology:SIP}
\label{sym:SIP}

``Session Initiation Protocol (SIP) is an application-layer control (signaling) protocol for creating, modifying, and terminating sessions with one or more participants. These sessions include Internet telephone calls, multimedia distribution, and multimedia conferences.''\cite{RFC3261}

SIP is the format of control signal in VoIP. It describes the sender, receiver. A agent use SIP messages to register on a proxy, establish session or close session. 

\subsection{SDP}
\label{sec:Solution:InvolvedTechnology:SDP}
\label{sym:SDP}

SDP is short for Session Description Protocol. ``It is intended for describing multimedia sessions for the purposes of session announcement, session invitation, and other forms of multimedia session initiation.''\cite{RFC4566}

A SIP message may carry a SDP message. The SDP message contains protocol version, session name, information, and most important, the connection data and media descriptions. This supplies a way to manipulate the connection of media flow. 

\subsection{Mobile Front Controller}
\label{sec:Solution:InovlvedTechnology:MobileFrontController}

``Mobile Front Controller (MFC) is a light-weight Java EE web application framework for creating web applications for web browsing and mobile browsing.'' \cite{MobileFrontController}



\subsection{Web Service}
\label{sec:Solution:InovlvedTechnology:WebService}











% ********** End of chapter **********
