
% ********** Chapter 8 **********
\chapter{Web Service Interface}
\label{sec:WebServiceInterface}

\section{Introduce web service and metro}

\begin{quote}
\textit{``A Web service is a software system designed to support interoperable machine-to-machine interaction over a network. It has an interface described in a machine-processable format (specifically WSDL). Other systems interact with the Web service in a manner prescribed by its description using SOAP-messages, typically conveyed using HTTP with an XML serialization in conjunction with other Web-related standards.''}\cite{WebServicesW3C}
\begin{flushright}
--W3C
\end{flushright}
\end{quote}

A web service interface of Web Call Example Application gives a most broad way for clients. Chapter \ref{sec:JavaMEClient} describes a Java ME Client of this web service interface. However, it is definitely not the only client for the web service. As long as a clients implements the interface, it can use the web call service and database. 

The web service interface of Web Call Example Application is based on \textsf{Metro}. Metro is a \textit{high-performance}, \textit{extensible}, \textit{easy-to-use} web service stack\cite{MetroHomepage}. It is supported by Sun Microsystems. For more detail about Metro, please refer to Metro Homepage at \href{https://metro.dev.java.net/}{https://metro.dev.java.net/}. 

\section{SOAP web service}

The web service source file is \\ \texttt{sip.components.webapp.webserviceg.WebCallImpl}. 

List \ref{WebServiceImplementation} shows a fragment of the source code that calls operation in Sip Call Component. 


\lstdefinelanguage{Java}
{morekeywords={import, @WebService, @WebMethod, @WebResult, @WebParam, public, class, String}}
\lstset{language=Java}

\begin{lstlisting}[frame=lines, float=!bph, caption=Web Service implementation (fragment), label=WebServiceImplementation]

...
import sip.components.core.controller.CallController;
...

@WebService(name = "WebCall")
public class WebCallImpl {

...
@WebMethod
@WebResult(name = "callID")
public String call(@WebParam(name = "clientA") String clientA,
               @WebParam(name = "clientB") String clientB,
               @WebParam(name = "sipAccountName") 
                         String sipAccountName,
               @WebParam(name = "implType") String implType,
               @WebParam(name = "username") String username,
               @WebParam(name = "password") String password)
                         Throws PersistenceLayerException, 
                         AuthenticationFailedException {
                             
...
...
CallController controller = cf.createAudioCallController();
controller.setClientA(clientA);
controller.setClientB(clientB);
controller.register();
controller.start();
...

}
\end{lstlisting}


The implementing class of web service uses the \texttt{@WebService} annotation to set the port name, service name and the target namespace. The web service methods use the \texttt{WebMethod} to annotate the operation name and actions.  The parameter of the method is annotated with \texttt{@WebParam}.

There are nine web service methods in the implementation Web Service:
\begin{itemize}
\item \texttt{getUserURIs:} Get the list of user's URIs.
\item \texttt{getContacts:} Get the contact list.
\item \texttt{getRecentCalls:} Get the recent calls list.
\item \texttt{getSipAccounts:} Get accounts for sip providers. The register of sip provider's accounts can be done at both desktop and mobile browser views.
\item \texttt{getImplTypes:} Get the implementation types list.
\item \texttt{call:} Establish phone calls. In this method, the user's phone number and destination phone number will also be stored in database. A Ring Buffer is used to store the recent calls.
\item \texttt{cancelCall:} cancel a phone call.
\item \texttt{getState:} get a phone call's state.
\item \texttt{syncContacts:} Synchronize contacts.  If same name with different phone number happens, the number in client will be kept.

\end{itemize}

For details about the method please refer the source code in \texttt{WebCallImpl}.

\section{Synchronize}

% ********** End of chapter **********
