% ********** Chapter 1 **********
\chapter{Background}
\label{sec:Background}


\section{Introduction}
\label{sec:Background:Introduction}

Voice over Internet Protocol (VoIP)\label{sym:VoIP} enables the communications between Internet users and the endpoints in PSTN circuit switched (CS)\label{sym:CS} networks. As we know, Session Initiation Protocol (SIP) is widely adopted as a signaling protocol for VoIP communications. Because of its simplicity, power and extensibility, it has also been selected by the Third Generation Partnership Project (3GPP)\label{sym:3GPP} as a major component of IP Multimedia Subsystem (IMS)\label{sym:IMS} for the evolved UMTS core network. 

As the world-leading supplier in telecommunications, Ericsson realizes that the position of VoIP technology in the future of telecom industry is absolutely outstanding. A project named Web Call Example Application(former \textit{Web Call SDK}) is planned by the department of Ericsson Developer Connection, to create an application that based on VoIP and make phone to phone calls. 

\section{Task}
\label{sec:Introduction:Task}

The task is to take all of the benefits of VoIP and create a powerful application for phone calls. The application should be simple, stable, reusable, extendable, integratable, easy to access and user friendly. 

It should be a splendid application that supplies the function for users to make phone call anytime, anywhere and to anyone in this world.



\section{Terminology}
\label{sec:Terminology}


\subsection*{Java}
\label{sec:Java}

The Java\texttrademark{} programming language is a popular high-level language which provides a portable feature and can be used on many different operating systems. 

The source code of Java is first written in plain text files which ends with the \nolinebreak\texttt{.java}. Then the Java source files are compiled into bytecodes which ends with \nolinebreak\texttt{.class} by Java compiler (javac). A \nolinebreak\texttt{.class} file is platform independent. It will be executed by the Java Virtual Machine\label{sym:JVM}\footnote{The terms ``Java Virtual Machine'' and ``JVM'' mean a Virtual Machine for the Java platform. \cite{TheJavaProgrammingLanguage}}.\cite{TheJavaProgrammingLanguage}

A Java application can be distributed in the format of Java ARchive (JAR\label{sym:JAR}) file.

\subsection*{Java EE}
\label{sec:JavaEE}
\label{sym:JavaEE}

Java Platform, Enterprise Edition (Java EE) is a set of coordinated technologies that significantly reduces the cost and complexity of developing, deploying, and managing multitier, server-centric applications.\cite{JavaEETechnology}


\subsection*{Java ME}
\label{sec:JavaME}
\label{sym:JavaME}

Java Platform, Micro Edition (Java ME) is a collection of technologies and specifications to create a platform that fits the requirements for mobile devices such as consumer products, embedded devices, and advanced mobile devices.\cite{JavaMETechnology}

\subsection*{MIDlet}
\label{sec:MIDlet}
\label{sym:MIDlet}

A MIDlet is a Java application framework for the Mobile Information Device Profile (MIDP\label{sym:MIDP}) that is typically implemented on a Java-enabled cell phone or other embedded device or emulator.

\subsection*{RMS}
\label{sec:RMS}
\label{sym:RMS}
The Java ME Record Management System (RMS) provides a mechanism through which MIDlets can persistently store data and retrieve it later.\cite{J2MERecordManagementStore}

\subsection*{JAD}
\label{sec:JAD}
\label{sym:JAD}

The Java Application Descriptor (JAD) file, as the name implies, describes a \textit{MIDlet} suite. The description includes the name of the MIDlet suite, the location and size of the JAR file, and the configuration and profile requirements. The file may also contain other attributes, defined by the Mobile Information Device Profile (MIDP), by the developer, or both.\cite{LearningPathMIDletLifeCycle}


\subsection*{SSL}
\label{sec:SSL}
\label{sym:SSL}

Secure Sockets Layer (SSL), is a cryptographic protocol which provides security and data integrity for communications over networks such as the Internet.\cite{SSL}


\subsection*{SIP}
\label{sec:SIP}
\label{sym:SIP}

Session Initiation Protocol (SIP) is an application-layer control (signaling) protocol for creating, modifying, and terminating sessions with one or more participants. These sessions include Internet telephone calls, multimedia distribution, and multimedia conferences.\cite{RFC3261}

SIP is the format of control signal in VoIP. It describes the sender, receiver. A agent use SIP messages to register on a proxy, establish session or close session. 

\subsection*{SDP}
\label{sec:SDP}
\label{sym:SDP}

SDP is short for Session Description Protocol. It is intended for describing multimedia sessions for the purposes of session announcement, session invitation, and other forms of multimedia session initiation.\cite{RFC4566}

A SIP message may carry a SDP message. The SDP message contains protocol version, session name, information, and most important, the connection data and media descriptions. This supplies a way to manipulate the connection of media flow. 

\subsection*{RTP}
\label{sec:RTP}
\label{sym:RTP}

RTP, the real-time transport protocol, provides end-to-end network transport functions suitable for applications transmitting real-time data, such as audio, video or simulation data, over multicast or unicast network services.\cite{RFC3550}

\subsection*{Mobile Front Controller}
\label{sec:MobileFrontController}
\label{MFC}

Mobile Front Controller (MFC) is a light-weight Java EE web application framework for creating web applications for web browsing and mobile browsing. \cite{MobileFrontController}

The mobile front controller uses a sevlet to handle http request, and redirect request to different kind of view. All views share a same logic. 

\subsection*{Web Service}
\label{sec:WebService}

A web service is defined by the W3C as ``a software system designed to support interoperable machine-to-machine interaction over a network''.\cite{WebServicesGlossary}

\subsection*{SOAP}
\label{sec:SOAP}
\label{sym:SOAP}

SOAP is a lightweight protocol intended for exchanging structured information in a decentralized, distributed environment. It uses XML technologies, an extensible messaging framework containing a message construct that can be exchanged over a variety of underlying protocols.\cite{SOAPVersion1dot2}

\subsection*{WSDL}
\label{sec:WSDL}
\label{sym:WSDL}

WSDL is an XML format for describing network services as a set of endpoints operating on messages containing either document-oriented or procedure-oriented information.\cite{WSDL1dot1} 

\subsection*{PSTN}
\label{sym:PSTN} 
\label{sec:PSTN}

PSTN is short for Public Switched Telephone Network. It is the network of the world's public circuit-switched telephone networks. In another word, it is just the traditional phone network.


\clearpage
\section{About}
\label{sec:Introduction:Background:About}

\textbf{Web Call Example Application} is an open source project at \href{http://www.ericsson.com/developer/}{Ericsson Developer Connection}\footnote{Ericsson Developer Connection (former Ericsson Mobility World Developer Porgram) is a department of Ericsson. It helps developers to create applications that incorporate telecommunication network capabilities, such as location-based services, charging, messaging and presence, with sustainability in mind.} (EDC), \href{http://www.ericsson.com/}{Ericsson}\footnote{Ericsson is a world-leading provider of telecommunications equipment and related services to mobile and fixed network operators globally.}. It is the successor project of \textbf{Web Call SDK}\cite{WebCallSDK} which is developed by Yuening Chen at Ericsson AB during the year 2007 and 2008. After Yuening finished Web Call SDK, the people at EDC found it is not stable enough and many of the functions are not usable. So they decided to start a new project follow Web Call SDK to make the it stable and add some new feature to it. The new project is called \textbf{Web Call Example Application}. As the main developer, I took over the new project at March 2008 and finished it at February 2009. I rewrite most code of the old project to make it stable and runnable and added some new feature to it. As a result, the \textbf{Web Call Example Application} is used as the base library of Ericsson's demo of \textbf{Using REST and Web Services to Mash Up Communications Capabilities}\cite{DemoAtJavaOne}\cite{EricssonJavaOne} at \href{http://java.sun.com/javaone/}{JavaOne}\texttrademark\footnote{JavaOne is an annual conference (since 1996) put on by Sun Microsystems to discuss Java technologies} 2009.
% ********** End of chapter **********
