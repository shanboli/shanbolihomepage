
\subsection*{Java}
\label{sec:Java}

The Java\texttrademark{} programming language is a popular high-level language which provides a portable feature and can be used on many different operating systems. 

The source code of Java is first written in plain text files which ends with the \nolinebreak\texttt{.java}. Then the Java source files are compiled into bytecodes which ends with \nolinebreak\texttt{.class} by Java compiler (javac). A \nolinebreak\texttt{.class} file is platform independent. It will be executed by the Java Virtual Machine\label{sym:JVM}\footnote{The terms ``Java Virtual Machine'' and ``JVM'' mean a Virtual Machine for the Java platform. \cite{TheJavaProgrammingLanguage}}.\cite{TheJavaProgrammingLanguage}

A Java application can be distributed in the format of Java ARchive (JAR\label{sym:JAR}) file.

\subsection*{Java EE}
\label{sec:JavaEE}
\label{sym:JavaEE}

Java Platform, Enterprise Edition (Java EE) is a set of coordinated technologies that significantly reduces the cost and complexity of developing, deploying, and managing multitier, server-centric applications.\cite{JavaEETechnology}


\subsection*{Java ME}
\label{sec:JavaME}
\label{sym:JavaME}

Java Platform, Micro Edition (Java ME) is a collection of technologies and specifications to create a platform that fits the requirements for mobile devices such as consumer products, embedded devices, and advanced mobile devices.\cite{JavaMETechnology}

\subsection*{MIDlet}
\label{sec:MIDlet}
\label{sym:MIDlet}

A MIDlet is a Java application framework for the Mobile Information Device Profile (MIDP\label{sym:MIDP}) that is typically implemented on a Java-enabled cell phone or other embedded device or emulator.

\subsection*{RMS}
\label{sec:RMS}
\label{sym:RMS}
The Java ME Record Management System (RMS) provides a mechanism through which MIDlets can persistently store data and retrieve it later.\cite{J2MERecordManagementStore}

\subsection*{JAD}
\label{sec:JAD}
\label{sym:JAD}

The Java Application Descriptor (JAD) file, as the name implies, describes a \textit{MIDlet} suite. The description includes the name of the MIDlet suite, the location and size of the JAR file, and the configuration and profile requirements. The file may also contain other attributes, defined by the Mobile Information Device Profile (MIDP), by the developer, or both.\cite{LearningPathMIDletLifeCycle}


\subsection*{SSL}
\label{sec:SSL}
\label{sym:SSL}

Secure Sockets Layer (SSL), is a cryptographic protocol which provides security and data integrity for communications over networks such as the Internet.\cite{SSLAtWiKi}


\subsection*{SIP}
\label{sec:SIP}
\label{sym:SIP}

Session Initiation Protocol (SIP) is an application-layer control (signaling) protocol for creating, modifying, and terminating sessions with one or more participants. These sessions include Internet telephone calls, multimedia distribution, and multimedia conferences.\cite{RFC3261}

SIP is the format of control signal in VoIP. It describes the sender, receiver. A agent use SIP messages to register on a proxy, establish session or close session. 

\subsection*{SDP}
\label{sec:SDP}
\label{sym:SDP}

SDP is short for Session Description Protocol. It is intended for describing multimedia sessions for the purposes of session announcement, session invitation, and other forms of multimedia session initiation.\cite{RFC4566}

A SIP message may carry a SDP message. The SDP message contains protocol version, session name, information, and most important, the connection data and media descriptions. This supplies a way to manipulate the connection of media flow. 

\subsection*{RTP}
\label{sec:RTP}
\label{sym:RTP}

RTP, the real-time transport protocol, provides end-to-end network transport functions suitable for applications transmitting real-time data, such as audio, video or simulation data, over multicast or unicast network services.\cite{RFC3550}

\subsection*{Web Service}
\label{sec:WebService}

A web service is defined by the W3C as ``a software system designed to support interoperable machine-to-machine interaction over a network''.\cite{WebServicesGlossary}

\subsection*{SOAP}
\label{sec:SOAP}
\label{sym:SOAP}

SOAP is a lightweight protocol intended for exchanging structured information in a decentralized, distributed environment. It uses XML technologies, an extensible messaging framework containing a message construct that can be exchanged over a variety of underlying protocols.\cite{SOAPVersion1dot2}

\subsection*{WSDL}
\label{sec:WSDL}
\label{sym:WSDL}

WSDL is an XML format for describing network services as a set of endpoints operating on messages containing either document-oriented or procedure-oriented information.\cite{WSDL1dot1} 

\subsection*{PSTN}
\label{sym:PSTN} 
\label{sec:PSTN}

PSTN is short for Public Switched Telephone Network. It is the network of the world's public circuit-switched telephone networks. In another word, it is just the traditional phone network.
