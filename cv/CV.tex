%%%%%%%%%%%%%%%%%%%%%%%%%%%%%%%%%%%%%%%%%%%%%%%%%%%%%%%%%%%%%%%%%%%%%%%%
%%%%%%%%%%%%%%%%%%%%%% Simple LaTeX CV Template %%%%%%%%%%%%%%%%%%%%%%%%
%%%%%%%%%%%%%%%%%%%%%%%%%%%%%%%%%%%%%%%%%%%%%%%%%%%%%%%%%%%%%%%%%%%%%%%%

%%%%%%%%%%%%%%%%%%%%%%%%%%%%%%%%%%%%%%%%%%%%%%%%%%%%%%%%%%%%%%%%%%%%%%%%
%% NOTE: If you find that it says                                     %%
%%                                                                    %%
%%                           1 of ??                                  %%
%%                                                                    %%
%% at the bottom of your first page, this means that the AUX file     %%
%% was not available when you ran LaTeX on this source. Simply RERUN  %% 
%% LaTeX to get the ``??'' replaced with the number of the last page  %% 
%% of the document. The AUX file will be generated on the first run   %%
%% of LaTeX and used on the second run to fill in all of the          %%
%% references.                                                        %%
%%%%%%%%%%%%%%%%%%%%%%%%%%%%%%%%%%%%%%%%%%%%%%%%%%%%%%%%%%%%%%%%%%%%%%%%

%%%%%%%%%%%%%%%%%%%%%%%%%%%% Document Setup %%%%%%%%%%%%%%%%%%%%%%%%%%%%

% Don't like 10pt? Try 11pt or 12pt
\documentclass[10pt]{article} 

% This is a helpful package that puts math inside length specifications
\usepackage{calc}
\usepackage{textcomp}
\usepackage{amssymb}

% Layout: Puts the section titles on left side of page
\reversemarginpar

%
%         PAPER SIZE, PAGE NUMBER, AND DOCUMENT LAYOUT NOTES:
%
% The next \usepackage line changes the layout for CV style section
% headings as marginal notes. It also sets up the paper size as either
% letter or A4. By default, letter was used. If A4 paper is desired,
% comment out the letterpaper lines and uncomment the a4paper lines.
%
% As you can see, the margin widths and section title widths can be
% easily adjusted.
%
% ALSO: Notice that the includefoot option can be commented OUT in order
% to put the PAGE NUMBER *IN* the bottom margin. This will make the
% effective text area larger.
%
% IF YOU WISH TO REMOVE THE ``of LASTPAGE'' next to each page number,
% see the note about the +LP and -LP lines below. Comment out the +LP
% and uncomment the -LP.
%
% IF YOU WISH TO REMOVE PAGE NUMBERS, be sure that the includefoot line
% is uncommented and ALSO uncomment the \pagestyle{empty} a few lines
% below.
%

%% Use these lines for letter-sized paper
\usepackage[paper=letterpaper,
            %includefoot, % Uncomment to put page number above margin
            marginparwidth=1.2in,     % Length of section titles
            marginparsep=.05in,       % Space between titles and text
            margin=1in,               % 1 inch margins
            includemp]{geometry}

%% Use these lines for A4-sized paper
%\usepackage[paper=a4paper,
%            %includefoot, % Uncomment to put page number above margin
%            marginparwidth=30.5mm,    % Length of section titles
%            marginparsep=1.5mm,       % Space between titles and text
%            margin=25mm,              % 25mm margins
%            includemp]{geometry}

%% More layout: Get rid of indenting throughout entire document
\setlength{\parindent}{0in}

%% This gives us fun enumeration environments. compactitem will be nice.
\usepackage{paralist}

%% Reference the last page in the page number
%
% NOTE: comment the +LP line and uncomment the -LP line to have page
%       numbers without the ``of ##'' last page reference)
%
% NOTE: uncomment the \pagestyle{empty} line to get rid of all page
%       numbers (make sure includefoot is commented out above)
%
\usepackage{fancyhdr,lastpage}
\pagestyle{fancy}
%\pagestyle{empty}      % Uncomment this to get rid of page numbers
\fancyhf{}\renewcommand{\headrulewidth}{0pt}
\fancyfootoffset{\marginparsep+\marginparwidth}
\newlength{\footpageshift}
\setlength{\footpageshift}
          {0.5\textwidth+0.5\marginparsep+0.5\marginparwidth-2in}
\lfoot{\hspace{\footpageshift}%
       \parbox{4in}{\, \hfill %
                    \arabic{page} of \protect\pageref*{LastPage} % +LP
%                    \arabic{page}                               % -LP
                    \hfill \,}}

% Finally, give us PDF bookmarks
\usepackage{color,hyperref}
\definecolor{darkblue}{rgb}{0.0,0.0,0.3}
\hypersetup{colorlinks,breaklinks,
            linkcolor=darkblue,urlcolor=darkblue,
            anchorcolor=darkblue,citecolor=darkblue}

%%%%%%%%%%%%%%%%%%%%%%%% End Document Setup %%%%%%%%%%%%%%%%%%%%%%%%%%%%


%%%%%%%%%%%%%%%%%%%%%%%%%%% Helper Commands %%%%%%%%%%%%%%%%%%%%%%%%%%%%
\newenvironment{narrow}[2]{%
\begin{list}{}{%
\setlength{\topsep}{0pt}%
\setlength{\leftmargin}{#1}%
\setlength{\rightmargin}{#2}%
\setlength{\listparindent}{\parindent}%
\setlength{\itemindent}{\parindent}%
\setlength{\parsep}{\parskip}%
}%
\item[]}{\end{list}} 

% The title (name) with a horizontal rule under it
%
% Usage: \makeheading{name}
%
% Place at top of document. It should be the first thing.
\newcommand{\makeheading}[1]%
        {\hspace*{-\marginparsep minus \marginparwidth}%
         \begin{minipage}[t]{\textwidth+\marginparwidth+\marginparsep}%
                {\large \bfseries #1}\\[-0.15\baselineskip]%
                 \rule{\columnwidth}{1pt}%
         \end{minipage}}

% The section headings
%
% Usage: \section{section name}
%
% Follow this section IMMEDIATELY with the first line of the section
% text. Do not put whitespace in between. That is, do this:
%
%       \section{My Information}
%       Here is my information.
%
% and NOT this:
%
%       \section{My Information}
%
%       Here is my information.
%
% Otherwise the top of the section header will not line up with the top
% of the section. Of course, using a single comment character (%) on
% empty lines allows for the function of the first example with the
% readability of the second example.
\renewcommand{\section}[2]%
        {\pagebreak[2]\vspace{1.3\baselineskip}%
         \phantomsection\addcontentsline{toc}{section}{#1}%
         \hspace{0in}%
         \marginpar{
         \raggedright \scshape #1}#2}


% An itemize-style list with lots of space between items
\newenvironment{outerlist}[1][\enskip\textbullet]%
        {\begin{itemize}[#1]}{\end{itemize}%
         \vspace{-.6\baselineskip}}

% An environment IDENTICAL to outerlist that has better pre-list spacing
% when used as the first thing in a \section 
\newenvironment{lonelist}[1][\enskip\textbullet]%
        {\vspace{-\baselineskip}\begin{list}{#1}{%
        \setlength{\partopsep}{0pt}%
        \setlength{\topsep}{0pt}}}
        {\end{list}\vspace{-.6\baselineskip}}

\newenvironment{king}
{\rule{1ex}{1ex}%
\hspace{\stretch{1}}}
{\hspace{\stretch{1}}%
\rule{1ex}{1ex}}


% An itemize-style list with little space between items
\newenvironment{innerlist}[1][\enskip\textbullet]%
        {\begin{compactitem}[#1]}{\end{compactitem}}

% To add some paragraph space between lines.
% This also tells LaTeX to preferably break a page on one of these gaps
% if there is a needed pagebreak nearby.
\newcommand{\blankline}{\quad\pagebreak[2]}

%%%%%%%%%%%%%%%%%%%%%%%% End Helper Commands %%%%%%%%%%%%%%%%%%%%%%%%%%%

%%%%%%%%%%%%%%%%%%%%%%%%% Begin CV Document %%%%%%%%%%%%%%%%%%%%%%%%%%%%

\begin{document}
\makeheading{Shanbo Li}

\section{Contact Information}
%
% NOTE: Mind where the & separators and \\ breaks are in the following
%       table.
%
% ALSO: \rcollength is the width of the right column of the table \textasciitilde
%       (adjust it to your liking; default is 1.85in).
%
\newlength{\rcollength}\setlength{\rcollength}{2.85in}%
%
\begin{tabular}[t]{@{}p{\textwidth-\rcollength}p{\rcollength}}
Rum 0738            & \textit{Tel:} 0046 70-4646157 \\
Hanstav\"{a}gen 51           & \textit{E-mail:}
\href{mailto:shanboli@gmail.com}{shanboli@gmail.com}\\
164 53 Kista, Sweden    
&\textit{Homepage:}\href{http://www.isk.kth.se/~shanbo/}{http://www.isk.kth.se/\textasciitilde shanbo}\\
\end{tabular}

\section{Nationality}
%
Chinese

\section{Data of birth}
%
1984-08-10

\section{Gender}
%
male

\section{Objective}
Position as software engineer focus on Web 2.0 and Java EE technologies. 

\section{Education}
%
\href{http://www.kth.se/}{\textbf{Royal Institute of Technology / Kungliga Tekniska H\"{o}gskolan (KTH)}},\\ 
\textbf{Stockholm, Sweden}
\begin{outerlist}
\item[] Master's Degree,\\ 
        {Software Engineering of Distributed Systems} \\
        expected graduation date: April 2009
\end{outerlist}
\blankline

\href{http://www.njtu.edu.cn/en/}{\textbf{Beijing Jiaotong University}},\\
\textbf{Beijing, China}
\begin{outerlist}
\item[] 4-year Bachelor's Degree in Engineering, \\
             {Computer Science}, September 2003 - June 2007
\end{outerlist}       

\section{Professional Projects}
%
\href{http://www.ericsson.com/}{\textbf{Ericsson AB}}, \href{http://www.ericsson.com/mobilityworld}{Developer Program Ericsson Mobility World} ,
Sweden
\begin{outerlist}

\item[] \textit{Java Programmer}%
        \hfill \textbf{March 2008 to February 2009}

\item[] {\textbf{Web Call SDK}}
\begin{outerlist}
\item \textit{Description:}\\
AJava EE web application for making VoIP calls. It supplies both browser client and Java ME client. 
\item \textit{Technologies:}\\
IMS, VoIP, SIP, SDP, 3PCC, Web Application, Web Service, MIDlet
\item \textit{API/Framework:}\\
Java EE, Metro, MjSIP, JMF, Mobile Front Controller, JSR\nobreak\space 172, JSR\nobreak\space 75,\\ Tomcat, MySQL
\end{outerlist}
\end{outerlist}
\blankline

\section{OpenSource Projects}
\href{http://code.google.com/p/scallope/}{\textbf{Scallope}} 
\begin{outerlist}
\item \texttt{Description:}
A application acts as a web client login to sip provider's web page and make VoIP calls. 
\item \texttt{API/Framework:}
Java, Apache common HTTP.
\end{outerlist}

\section{Academic Projects}
\textbf{Bouncer}, KTH, Sweden \hfill \textbf{March 2008 to May 2008}
\begin{innerlist}
\item \texttt{Description:}
A Packet bouncer that "bounces" connections from one machine to another.
\item \texttt{API/Framework:}
Java, Jpcap
\end{innerlist}

\blankline

\textbf{Sip Speaker}, KTH, Sweden \hfill \textbf{March 2008 to May 2008}
\begin{outerlist}
\item \texttt{Description:}
A SIP user agent (UA) that waits for incoming calls and answers them when received.
\item \texttt{API/Framework:}
Java, JMF, FreeTTS
\end{outerlist}

\blankline

\textbf{Web Mail}, KTH, Sweden \hfill \textbf{March 2008 to May 2008}
\begin{outerlist}
\item \texttt{Description:}
A web server for sending e-mails. No 3rd party library.
\item \texttt{API/Framework:}
Java
\end{outerlist}

\blankline

\textbf{Flight Ticket Reservation}, KTH, Sweden \hfill \textbf{January 2008 to March 2008}
\begin{outerlist}
\item \texttt{Description:}
A web service for booking flight tickets.
\item \texttt{API/Framework:}
Java EE, JSF, Web Service (Metro), BPEL, JPA, MySQL 
\end{outerlist}

\blankline

\textbf{E-Shop}, KTH, Sweden \hfill \textbf{November 2007 to January 2008}
\begin{outerlist}
\item \texttt{Description:}
A web shop
\item \texttt{API/Framework:}
Java EE, Struts, MySQL
\end{outerlist}

\blankline

\textbf{Forex}, KTH, Sweden \hfill \textbf{November 2007 to January 2008}
\begin{outerlist}
\item \texttt{Description:}
A web site for checking currency exchange rate
\item \texttt{API/Framework:}
Java EE, JSF, EJB3
\end{outerlist}

\blankline

\textbf{Hangman}, KTH, Sweden \hfill \textbf{November 2007 to January 2008}
\begin{outerlist}
\item \texttt{Description:}
A Java ME game with a server.
\item \texttt{API/Framework:}
Java ME, Java SE, Swing
\end{outerlist}

\blankline

\textbf{iAd}, KTH, Sweden \hfill \textbf{November 2007 to January 2008}
\begin{outerlist}
\item \texttt{Description:}
A resource management system for advertisment company, which is developed by the method of Extreme Programming.
\item \texttt{API/Framework:}
Ruby on Rails, MySQL
\end{outerlist}

\blankline

\textbf{Anti-Counterfeiting System}, BJTU, China \hfill \textbf{September 2006 to January 2007}
\begin{outerlist}
\item \texttt{Description:}
A online anti-conterfeiting system
\item \texttt{API/Framework:}
Java, Struts, Spring, Hibernate, MySQL
\end{outerlist}


\section{Social Skills}
{\textbf{Full of Team Spirit}}\\
{\textbf{Good Ability of Communication}}\\
{\textbf{Cope with Pressure}}\\
{\textbf{Willing to Learn New Technology}}\\
{\textbf{Good at Written and Spoken English (CET-6)}}\\
{\textbf{Travelling 7 days a week is Acceptable}}\\

\section{Knowledge Domain}
{\textbf{IP Multimedia Subsystem (IMS)}}\\ 
{\textbf{SIP}}, {\textbf{SDP}}, {\textbf{VoIP}}\\ 
{\textbf{TCP, UDP, HTTP}}\\ 
{\textbf{Network Security}}\\ 

\section{Technical Skills} 
%
{\textbf{6 Years of Java Programming Experience:}}\\
Very Good at Java EE/ Java SE/ Java ME

\blankline

{\textbf{Database development and Management:}}\\
JDBC, SQL, MySQL, Oracle

\blankline

{\textbf{Java API/Frameworks:}}\\
Hibernate, EJB3, Spring2.5, JSF, Struts, Swing, Apache\nobreak\space commons\nobreak\space logging,\\ Log4j, JUnit4 

\blankline

{\textbf{Skill set of Web Services development:}}\\
SOAP, Metro, AXIS, RESTful Web Service, BPEL, jUDDI


\blankline


{\textbf{Skill set of Web Site development:}}\\
XHTML, XHTML MP, HTML, CSS, Javascript, JSON, AJAX, JSP, JSF, XML,\\ Ruby \nobreak on \nobreak Rails, PHP 

\blankline

{\textbf{Application Server:}}\\
Apache Tomcat, Glassfish, JBoss, Apache HTTP Server

\blankline

{\textbf{Software Development Concept:}}\\
Object-oriented Programming, UML, Design \nolinebreak Patterns
, Scrum, Extreme Programming,\\ Agile Software Development

\blankline

{\textbf{Project management and comprehension tool:}}\\
Ant, Maven

\blankline

{\textbf{Source Code Management:}}\\
Subversion, CVS

\blankline

{\textbf{Development Tools:}}\\
Eclipse, Intellij IDEA, NetBeans, IBM\circledR\space Rational\circledR\space Software Architect (RSA), JBuilder

\blankline

{\textbf{Other Programming Experience:}}\\
Ruby/Ruby on Rails, C/C++, PHP, Erlang, Prolog

\blankline

{\textbf{Documentation Tools:}}\\ 
\LaTeX{}, Microsoft Office,
and other common productivity packages for Windows,\\  Mac\nolinebreak OS\nolinebreak X, and
Linux platforms

\blankline

{\textbf{Operating Systems:}}\\
Windows, Mac OS X, Linux

\newpage
\section{References}
\begin{minipage}[t]{0.9\textwidth}
\begin{itemize}
\item Peter Yeung \\ Chief Strategist - Strategy \& Development \\ Developer Program, Ericsson Mobility World \\ Ericsson \\
  +46 1071 36652\\
  \href{mailto:peter.yeung@ericsson.com}{\tt peter.yeung@ericsson.com}
\end{itemize}
\begin{itemize}
\item Mihhail Matskin, PhD \\ Professor of Software Engineering  \\ Royal Institute of Technology (KTH), Sweden \\
  +46 8 790 4128 \\
  \href{mailto:misha@kth.se}{\tt misha@kth.se}
\end{itemize}
\begin{itemize}
\item Nima Dokoohaki \\ Researcher  \\ Royal Institute of Technology (KTH), Sweden \\
  +46 8 790 4149 \\
  \href{mailto:nimad@kth.se}{\tt nimad@kth.se} \vspace{2in}
\end{itemize}
\end{minipage}

% Footer

\begin{narrow}{0.65in}{2.0in}
\begin{center}
\begin{footnotesize}
Last updated: \today \\
\href{http://www.isk.kth.se/~shanbo/CV.pdf}{\tt http://www.isk.kth.se/\textasciitilde shanbo/CV.pdf}
\end{footnotesize}
\end{center}
\end{narrow}


\end{document}

%%%%%%%%%%%%%%%%%%%%%%%%%% End CV Document %%%%%%%%%%%%%%%%%%%%%%%%%%%%%
